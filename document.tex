\documentclass[12pt]{report}

\usepackage[version=4]{mhchem}
\usepackage[latin1]{inputenc} % acentuação 
\usepackage[brazil]{babel} % tradução pt-br
\usepackage{indentfirst} 


\begin{document}
	\chapter*{\begin{center}
			\textbf{RESUMO}
	\end{center}}

	O desenvolvimento do presente trabalho foi realizado primeiramente com a síntese dos nanocompósitos entre \ce{rGO/NiMo} pelo método poliol com adições de água, que variaram entre 0 e 1 mL. Os matérias produzidos foram submetidos a tratamento térmico em atmosfera inerte e então caracterizados por diferentes técnicas. A produção de filmes finos sobre substrato de FTO (óxido de estanho dopado com flúor, do inglês flurine tin oxide) foi realizado para caracterização eletroquímica e aplicação como baterias alcalinas e células a combustível.\\
	A síntese dos nanocompósitos foi realizada utilizando o método poliol a uma temperatura de 180 °C. Os nanocompósitos foram obtidos através da redispersão de óxido de grafeno (GO, do inglês graphene oxide) em etileno glicol, preparado previamente através da esfoliação do óxido de grafite (Gr-O, do inglês graphite oxide) preparado pelo método de Hummer’s modificado. Os percursores metálicos utilizados foram acetato de níquel e molibdato de sódio (Na2MoO4 · 2H2O) em uma proporção mássica Ni:Mo de 1:1. Com o intuito de aumentar a retenção de Mo nos nanocompósitos, também foram realizadas sínteses adicionando os volumes de 100, 500 e 1000 μL de água ultrapura logo após a adição dos precursores metálicos. Todas as metodologias levaram a formação de α-Ni(OH)2 com a redução de GO para óxido de grafeno reduzido (rGO, do inglês reduced graphene oxide). Os resultados obtidos até o momento sugerem a incorporação de Mo na estrutura cristalina do Ni(OH)2 devido à ausência de picos atribuídos apenas a possíveis compostos de Mo como óxidos e hidróxidos.\\
	O estudo do tratamento térmico teve o propósito da redução térmica dos constituintes metálicos e melhora de qualidade estrutural do rGO. As temperaturas foram escolhidas através de análises de termogravimetria (TGA, do inglês Thermogravimetric analysis), sendo elas 240, 420 e 600 °C, em atmosfera inerte de nitrogénio (N2). Como resultado, foi possível observar a redução da massa dos materiais submetidos ao tratamento em 240 °C, com a mudança significativa do tamanho particula de 25 para 12 nm. A mudança de composição para níquel metálico com estrutura cristalina cúbica de face centrada (Nicfc) ocorreu no tratamento com as temperaturas de 420 e 600 °C, sendo obsevado um tamanho médio de particula igual a 7,7 e 12 nm, respectivamente.\\
	As amostras foram utilizadas para confecção de elétrodos através do gotejamento de uma redispersão acima de substratos de FTO, e caracterizados em sistema de 3 elétrodos em meio alcalino de NaOH 1 mol mL-1 utilizando analises de voltametria cíclica e carga-descarga. Os resultados demonstraram que pode-se chegar a valores de capacaidade na casa dos 109,4 mA h g-1 para os filmes de rGO/NiMo(OH)2-0,5mLH2O a uma corrente específica de 1,43 A g-1.\\
	Os estudos para a aplicação como células a combustivel estão sendo feitos pelo prof. Dr. Paulo J. S. Maia do Grupo de Eletrocatálise e Química Bioinorgânica, localizado na Universidade Federal do Rio de Janeiro, Campus Macaé. \pagebreak
	
\end{document}